In questa tesi si assumeranno per scontate le nozioni basilari di sequenza, fattore e fattorizzazione, parola, stringa, alfabeto. Inoltre, si assumerà per scontata l'operazione di differenza simmetrica fra due stringhe.

\section{Absent Words, Minimal Absent Words}

Una \textit{absent word} o \textit{parola assente} (detta talvolta \textit{forbidden word} o \textit{parola proibita}) di una certa sequenza $S$ è tale se essa non occorre nella sequenza $S$ stessa. 

Le \textit{absent words} rappresentano, in effetti, la tipologia più genuina di informazione negativa: costituiscono, infatti, un dato che non si presenta in nessuna forma o composizione nella sequenza in esame. In generale, se la sequenza $S$ ha lunghezza $n$, il numero di \textit{absent words} è al più esponenziale in $n$. 

\vspace{3mm}

E' altresì possibile individuare specifiche classi di \textit{absent words}, quali le \textit{minimal absent words}, il cui numero è lineare in $n$. Le \textit{minimal absent words} sono \textit{absent words} di una sequenza $S$ che non occorrono propriamente nella sequenza, ma di cui - invece - tutte le fattorizzazioni proprie occorrono nella sequenza. Si è dunque davanti ad un contrasto interessante: da un lato informazione negativa cruda e grezza; dall'altro, informazione positiva derivata dalla fattorizzazione di informazione negativa. 

\(M_L = \{aub | a,b \in \Sigma*, aub\not\in L, au,ub\in L\} \)

Lo studio di queste fattorizzazioni e la loro applicazione pratica nell'implementazione di \verb|scMAW| ci permetterà di stabilire un metodo efficiente per confrontare stringhe impiegando, alla base, informazione genuinamente negativa.

\vspace{3mm}

Risulta dunque immediato osservare che, impiegando le \textit{minimal absent words} di sequenze $S_1$ e $S_2$, è possibile determinarne la similiarità in tempo proporzionale rispetto alla loro lunghezza $n_1$ e $n_2$, a patto che le due sequenze appartengano al medesimo alfabeto. In particolare, studieremo l'applicazione delle\textit{ minimal absent words} nella cornice dell'alfabeto del genoma $\Sigma_{DNA} =\{A, C, G, T\}$.

\vspace{10mm}

\underline{\textbf{Risultati basati sulle Minimal Absent Words}}

\vspace{5mm}

In letteratura, un moderato numero di risultati sono stati prodotti a partire dalla nozione di \textit{minimal absent words}. Si tratta di soluzioni di complessità, generalmente, $O(n)$, sia in termini di spazio che di tempo, posto e fissato un alfabeto preciso, di lunghezza ben definita. Tuttavia, è interressante osservare la moltitudine di strutture dati impiegate nell'implementazione di questi algoritmi: suffix arrays, suffix trees, suffix automata \cite{CROCHEMORE1998111}, per citarne alcuni. Alcune soluzioni risultano meno performanti di altre, presentando una complessità (nel caso peggiore) pari a $O(n^2 )$ \cite{findingMaw}.

\vspace{3mm}

In questa tesi, presenteremo e analizzeremo - nel Capitolo 3 - l'algoritmo \verb|scMAW|, uno dei più recenti risultati nell'ambito di confronto fra stringhe impiegando le \textit{minimal absent words} e di soddisfacente complessità, sia in termini di tempo che di spazio.

\section{Suffix Arrays}

Come osservato, è possibile introdurre il concetto di minimal absent words impiegando le strutture dati più disparate, come gli automi e i suffix arrays. Questi ultimi sono centrali nell'implementazione dell'algoritmo in esame, quale \verb|scMAW|.

\vspace{3mm}

I suffix arrays, risultato originale del 1990 di Manber-Myers \cite{suffixArray}, sono un array ordinato di tutti i suffissi di una stringa $S$. E' una struttura dati ampiamente utilizzata nell'indicizzazione di testi, negli algoritmi di compressione e nel campo della bibliometrica. La loro complessità è, sia in media che nei casi peggiori, $O(n)$, risultando dunque una struttura dati efficiente e di semplice costruzione.

\vspace{3mm}

In generale, posta $S=S[1]S[2]...S[n]$ una stringa di lunghezza $n$ denotata dai suoi fattori monocarattere, il suffix array $A$ di $S$ è definito come array di interi costituito dalle posizioni iniziali dei suffissi S in ordine lessicografico. $A[i]$ conterrà, in effetti, la posizione iniziale dell'$i$-esimo suffisso più piccolo in $S$. Ogni suffisso compare esattamente una volta. Didatticamente, viene generalmente utilizzata la stringa $banana$ per fornire un esempio.

\vspace{3mm}

Si ponga $S=banana$. Risulta che $S[1]=b, S[2]=a, S[3]=n, ..., S[6]=a, S[7]=blank$. Il $blank$ è generalmente indicato con un trattino basso.

\vspace{3mm}

Si considerino i suffissi di $S$.

\begin{table}[H]
\centering
\begin{tabular}{ll}
\textbf{Suffisso} & \textbf{i} \\
\textit{banana\_} & \textbf{1} \\
\textit{anana\_}  & \textbf{2} \\
\textit{nana\_}   & \textbf{3} \\
\textit{ana\_}    & \textbf{4} \\
\textit{na\_}     & \textbf{5} \\
\textit{a\_}      & \textbf{6} \\
\textit{\_}       & \textbf{7}
\end{tabular}
\end{table}

Ordiniamo i suffissi in ordine lessicografico mantenendo gli indici attuali.

\begin{table}[H]
\centering
\begin{tabular}{ll}
\textbf{Suffisso} & \textbf{i} \\
\textit{\_}       & \textbf{7} \\
\textit{a\_}      & \textbf{6} \\
\textit{ana\_}    & \textbf{4} \\
\textit{anana\_}  & \textbf{2} \\
\textit{banana\_} & \textbf{1} \\
\textit{na\_}     & \textbf{5} \\
\textit{nana\_}   & \textbf{3}
\end{tabular}
\end{table}

Il suffix array $A$ associato conterrà, in maniera associativa, gli indici $i$ soprastanti. Risulterà, dunque, $A[1]=7, A[2]=6, A[3]=4, ..., A[7]=3$.

\vspace{3mm}

E' altrettanto possibile costruire un suffix tree in funzione dei suffix arrays. La costruzione dell'albero è, tuttavia, soggetto alla specificità dell'algoritmo che lo sfrutta.

\section{Range Minimum Queries}

Una range minimum query \cite{RMQ} permettere di trovare il valore minimo in una sottostringa di un array di oggetti confrontabili (nel nostro caso, numeri interi). Sono generalmente utilizzati lateralmente agli LCP (Longest Common Prefix) arrays, quando necessario.

\vspace{3mm}

Dato un array $A[1...n]$ di $n$ oggetti confrontabili e ordinati, il range minimum query $RMQ_A (l,r)$ coincide con $arg_{min} A[k]$, con $1\leq l\leq k \leq r \leq n$. La funzione restituisce la posizione dell'elemento più piccolo nel sotto-array determinato da $A[l...r]$. 

\vspace{3mm}

Ad esempio, per l'array $A=0,5,2,5,4,3,1,6,3$, risulterà che il range minimum query di $A[3...8]=2,5,4,3,1,6$ è 7, poiché $A[7]=1$ ne é il valore più piccolo, dunque il minimo, ottenuto tramite confronti più o meno lineari.

\vspace{3mm}

La nozione di range minimum queries è, analogamente a quella di suffix arrays, centrale alla costruzione dell'algoritmo \verb|scMAW|.

\section{Metrica LW di Crochemore}

Chairungsee e Crochemore del King's College di Londra hanno presentato, nel 2012, una misura di similiarità fra due stringhe $x$ e $y$ impiegando le \textit{minimal absent words} \cite{CHAIRUNGSEE2012109} di ciascuna delle due, rispettivamente $M^l_x$ e $M^l_y$, dove $l$ è la lunghezza massima fra le due stringhe.

\vspace{3mm}

La metrica, che restituisce un float, è calcolata come sommatoria di $\frac{1}{|w|^2}$ dove ogni $w$ è una parola ottenuta dalla differenza simmetrica fra le minima absent words di $x$ e di $y$.

\[LW_l (x, y)=\sum_{w \in M^l_x \Delta M^l_y} \frac{1}{|w|^2 }\]

In generale, nell'implementare l'algoritmo \verb|scMAW| considereremo una più generica metrica LW che non tiene conto della lunghezza $l$.

\[LW (x, y)=\sum_{w \in M_x \Delta M_y} \frac{1}{|w|^2 }\]

Ad esempio, posto $x=abaab$ e $y=aabbbaa$, risulterà che $M_x=aaa, aaba, bab, bb$ (osserviamo che tutte le fattorizzazioni di $M_x$ sono presenti in $x$) e $M_y=aaa,bbbb,aba,abba,bab,baab$.

\vspace{3mm}

Risulterà:

\vspace{3mm}

\(M_x \Delta M_y ={aaba,aba,abba,baab,bb,bbbb}\)

\vspace{3mm}

Di consequenza, il soprastante risultato verrà dato in pasto alla sommatoria produttrice della metrica LW. Ogni lunghezza (inversa, ed elevata al quadrato) degli elementi dell'insieme ($aaba$, $aba$, ...) sarà sommata, al fine di produrre la metrica.

\(LW(x,y) = \frac{1}{4^2}+\frac{1}{3^2}+\frac{1}{4^2}+\frac{1}{4^2}+\frac{1}{2^2}+\frac{1}{4^2}\)

Questa misura permette di quantificare la distanza fra due parole, definendone la similitudine. Risulta, infatti, $LW(x,x)=0$. 