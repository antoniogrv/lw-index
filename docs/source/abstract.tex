I confronti fra stringhe sono un campo applicativo critico nell'informatica e, in particolare, nella bioinformatica. In letteratura, nel corso dei decenni sono state proposte numerose soluzioni nella forma di algoritmi più o meno efficienti in grado di misurare in maniera precisa il grado di similitudine fra un insieme di stringhe. Tuttavia, ci si pone il problema di elaborare un algoritmo di confronto fra stringhe che, invece di computare in funzione dell'informazione positiva - ottenuta analizzando la composizione propria e reale delle stringhe in input - sfrutti piuttosto l'informazione negativa intrinseca delle stesse nella forma delle loro \textit{minimal absent words}. Successivamente, si analizza lo stato dell'arte dei tool visuali adibiti alla generazione di grafici illustrativi riguardanti i medesimi confronti, e si propone un'alternativa adatta alle speciali esigenze in materia di studio, sviluppando un tool web in grado di accogliere, integrare e computare algoritmi basati sull'informazione negativa in virtù di un'opportuna metrica, e dunque capace di elaborare uno spettro di rappresentazioni grafiche che consentano di effettuare valutazioni qualitative degli algoritmi impiegati.