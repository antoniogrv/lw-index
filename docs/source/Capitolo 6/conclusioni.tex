Questo lavoro di tesi avevo lo scopo di esplorare, ad alto livello, le potenzialità dell'informazione negativa nell'ambito della bioinformatica, ed in particolare nella cornice del più ampio campo di lavori circa il confronto fra sequenze del genoma, benché sia applicabile - in generale - a stringhe su alfabeti ben definiti, come potrebbe essere quello proteico. 

Successivamente, il lavoro di tesi ha imposto lo studio approfondito di un algoritmo che facesse dell'informazione negativa, nella forma delle \textit{absent words} prima, e delle \textit{minimal absent words} poi, la nozione matematica critica nella sua composizione algoritmica. 

Lo studio ha dunque condotto al reverse engineering e alla comprensione generale del funzionamento del tool SMART che, dopo un'attenta revisione della natura dell'applicativo e delle modalità di elaborazione dei dati in input e ouput che adottasse, è risultato incompatibile con l'obiettivo generale di questo lavoro di tesi, il quale si prefissava - in origine - di integrare algoritmi come \verb|scMAW| in un tool di comparazione qualitativa.

\vspace{3mm}

Poiché SMART non è risultato adatto ai nostri fini, i lavori di tesi sono stati ridirenzionati, mutando in parte. Dato che non esisteva, al meglio delle conoscenze dell'autore di questa tesi,  e al momento della stesura di quest'ultima, un tool simil-SMART in grado di accogliere algoritmi che sfruttano l'informazione negativa e siano implementati nelle specifiche modalità indicate nei capitoli precedenti, si è deciso di realizzare un tool ad hoc, del tutto originale, adottante un'architettura client-server e implementato con tecnologie moderne e all'avanguardia. Si tratta di una one-page web-application che soddisfa a pieno i requisiti funzionali previsti dal progetto originario della tesi di laurea.

\vspace{3mm}

Risulta evidente che il tool oggetto di sviluppo sia ampiamente arricchibile con nuove funzionalità, e che sia sufficientemente versatile da accogliere un importante numero di algoritmi di confronto fra stringhe. Risulta altresì evidente che il tool possa essere notevolmente migliorato, soprattutto sul fronte dell'usabilità; d'altro canto, un possibile sviluppo futuro potrebbe essere rappresentato, similmente a SMART, dalla realizzazione di una controparte desktop, e dunque non basata su web-browser, dell'applicativo. Poiché il software tratta un ambito di così ampio respiro come quello dei confronti fra stringhe, è banale evidenziare come le modifiche apportabili all'implementazione - funzionale e non - siano innumerevoli (ad esempio, si potrebbe permettere all'utente di immettere direttamente file FASTA precompilati).

\vspace{3mm}

Il lavoro di tesi ha, in conclusione, evidenziato le potenzialità dell'informazione negativa nel quadro della bioinformatica, e come la questione del calcolo della similitudine fra stringhe sia, ad oggi, un argomento ancora ricco di spunti di riflessione e innovazioni scientifiche.