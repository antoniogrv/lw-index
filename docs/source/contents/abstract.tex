\clearpage
\setcounter{page}{1}

{\Huge \textbf{Abstract}}

\vspace{15mm}

Le stringhe sono un potente strumento matematico che si confà ad un spettro ampio e versatile di problematiche, fra cui la bioinformatica. I confronti fra stringhe, fra le altre cose, sono attori centrali nell'evidenziare similiarità delle sequenze genomiche in termini dei loro pattern costituenti.

In letteratura, la maggior parte delle soluzioni algoritmiche al problema dei confronti fra stringhe risultano computazionalmente onerose e prevedono tecniche di confronto allineate; d'altro canto, recenti considerazioni accademiche hanno evidenziato le potenzialità dell'informazione negativa, intesa in questa tesi come \textit{absent words}, nel contesto di confronti alignment-free.

La tesi ha l'obiettivo di integrare un algoritmo alignment-free di confronti fra sequenze che fa uso di \textit{absent words} nel tool String Matching Algorithms Research Tool (SMART), al fine di dimostrare la qualità del risultato.

Forniamo dunque gli elementi matematici fondamentali per la comprensione dell'algoritmo in esame; né elaboriamo, quindi, una codifica prima in pseudocodice e poi in linguaggio di programmazione C; e, infine, procediamo ad integrare il suddetto algoritmo nel tool SMART per ottenere una visione completa della sua resa qualitativa rispetto ad altre soluzioni ben note.