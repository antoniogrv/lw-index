Il confronto fra stringhe è, in generale, un campo di ricerca di largo interesse da parte della comunità scientifica internazionale. Le stringhe risultano una codifica comune per un vasto spettro di oggetti che possono vantare proprietà di confrontabilità. Fra questi, il genoma rappresenta uno dei soggetti  principali dei confronti fra stringhe, essendo banalmente codificabile su un alfabeto ben definito, quale l'alfabeto del genoma. L'analisi del genoma è legata strettamente ai confronti con altre sequenze genomiche, motivo per cui la genomica computazionale ne fa una delle sue principali materie di studio. In generale, la codifica in stringa del genoma permette di individuarne caratteristiche qualitative e di rilevare similiarità con filamenti differenti, risultando dunque centrale in ambiti di ricerca internazionali e non.

\vspace{3mm}

In particolare, in questa tesi viene affrontanto il problema degli algoritmi di confronto fra stringhe su alfabeti del genoma con implementazioni \textit{non-allineate} (o \textit{alignment-free}), e cioè che non manipolano la composizione propria della stringa genomica, al fine di ottenere una risultato quanto più efficiente possibile. D'altro canto, il lavoro di tesi riguarda per lo più l'ambito della comparazione qualitativa fra questi algoritmi di confronto fra stringhe; vengono dunque esplorati tool, più o meno recenti, che permettono di effettuare considerazioni di questa natura. Tuttavia, l'assenza di tool significativamente compatibili con gli algoritmi in esame porterà alla realizzazione di un applicativo ad hoc per le necessità evidenziate. 

\vspace{3mm}

Nel Capitolo 2 si introducono nozioni matematiche e cenni teorici fondamentali alla successiva comprensione degli argomenti trattati. Fra le nozioni in esame, le più importanti sono banalmente le minimal absent words e la metrica LW, entrambe centrali sia nell'implementazione dell'algoritmo in esame che del tool oggetto di sviluppo; vengono altresì esplorate strutture dati efficienti impiegate in numerose applicazioni nel campo della bioinformatica, come i suffix arrays.

Nel Capitolo 3 si analizza con precisione il funzionamento e parte dell'implementazione dell'algoritmo \verb|scMAW|, recente risultato originale di un gruppo di studio che propone una soluzione efficiente circa i confronti fra stringhe impiegando le minimal absent words. Si forniscono dettagli sulla costruzione delle strutture matematiche impiegate e sulle modalità di immissione dell'input e restituzione dell'output.

Nel Capitolo 4 si analizza il tool SMART, ideato per ospitare algoritmi di confronto fra stringhe e, in particolare, per individuare il numero di occorrenze di specifici pattern all'interno di testi più o meno voluminosi. Si osserveranno, in particolare, le cause per le quali il tool SMART risulta incompatibile e inoperabile con la specificità degli algoritmi in esame.

Nel Capitolo 5, verrà delineata la realizzazione e lo sviluppo del tool LW Index, che rappresenta il risultato originale di questa tesi di laurea. Verranno fornite specifiche circa i requisiti funzionali dell'applicativo, le scelte progettuali e architettoriali, la selezione delle tecnologie; verrà, inoltre, fornita una panoramica completa sui più importanti componenti dell'implementazione del tool, arricchendo l'esposizione con porzioni di codice ed esempi illustrativi.

Nel Capitolo 6, si effettuano diverse considerazioni sul lavoro svolto e si delineano alcuni possibili sviluppi futuri circa il tool realizzato.