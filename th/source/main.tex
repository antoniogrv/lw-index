\documentclass{article}
\usepackage[utf8]{inputenc}
\usepackage[a4paper, total={5.5in, 7.5in}]{geometry}
\usepackage{graphicx}
\usepackage{amsmath}
\renewcommand{\baselinestretch}{1.75}
\renewcommand{\contentsname}{Indice}
\usepackage{xcolor}
\usepackage{changepage}
\usepackage{lipsum}
\usepackage{setspace}
\usepackage{multicol}
\usepackage{etoolbox}
\usepackage[explicit]{titlesec}
\thispagestyle{empty}

\begin{document}

    \begin{center}
    
        \includegraphics[scale=0.2]{assets/logo_standard.png}
    
        \vspace{2mm}
    
        \textsc{\huge Università degli Studi di Salerno}
        
        \vspace{5mm}
        
        {\LARGE Dipartimento di Informatica}
        
        {\LARGE Corso di Laurea Triennale in Informatica}
        
        \vspace{10mm}
        
        \textsc{\huge Tesi di Laurea}
        
        \vspace{10mm}
        
        \textbf{\huge Il ruolo dell'informazione negativa nei confronti alignment-free fra stringhe: integrazione di un algoritmo di Minimal Absent Words nel tool SMART}
        
    \end{center}
    
        \vspace{20mm}
    
        {\large \setstretch{1.5}
            \begin{multicols}{2}
                \begin{adjustwidth}{0pt}{0pt}\textsc{Relatrice}\end{adjustwidth}
                Prof.ssa \textbf{Rosalba Zizza}\\
                Università degli Studi di Salerno
                
                \columnbreak
                
                \hspace*{\fill}\textsc{Candidato}\\
                \hspace*{\fill}\textbf{Antonio Gravino}\\
                \hspace*{\fill}Matricola: 05121 07161
            \end{multicols}
        }
        
        \begin{center}
            \vspace*{\fill}Anno Accademico 2021-2022
        \end{center}
        
        \break
        
        \thispagestyle{empty}

\hspace*{\fill}\textit{Alla mia famiglia,}

\vspace{-2mm}

\hspace*{\fill}\textit{faro in un oceano di incertezze.}

\vspace{5mm}

\hspace*{\fill}\textit{Ai miei colleghi universitari,}

\vspace{-2mm}

\hspace*{\fill}\textit{esempi di vigore e costanza.}

\vspace{5mm}

\hspace*{\fill}\textit{A me,}

\vspace{-2mm}

\hspace*{\fill}\textit{per aver perseverato nonostante le avversità.}

\vspace{5mm}

\hspace*{\fill}\textit{E infine al popolo ucraino,}

\vspace{-2mm}

\hspace*{\fill}\textit{che davanti ad un nemico apparentemente invincibile,}

\vspace{-2mm}

\hspace*{\fill}\textit{ha dato dimostrazione di grande forza e risolutezza,}

\vspace{-2mm}

\hspace*{\fill}\textit{rinnovando profondamente il mio spirito europeista.}

\vspace{5mm}
        
        \clearpage
\setcounter{page}{1}

{\Huge \textbf{Abstract}}

\vspace{15mm}

Lorem ipsum dolor sit amet, consectetur adipiscing elit. Nunc nec dapibus tortor. Fusce sit amet turpis in urna ultrices aliquam sodales et massa. Fusce tellus urna, malesuada vitae posuere at, porta pharetra ipsum. Nam auctor ex vel justo luctus aliquam. Etiam et ultricies augue. Integer et tellus velit. In auctor lobortis varius.
        
        \vspace{-20mm}
        
        \tableofcontents

        \section{Introduzione}

[..]

Questo primo capitolo consiste esclusivamente in una sintetica introduzione del lavoro svolto. Nel Capitolo Due, vengono presentate le componentistiche matematiche, con opportuni riferimenti accademici, utilizzate dall'algoritmo di confronto fra stringhe in esame, o dal tool SMART. Nel Capitolo Tre, si fornisce - a linee generali - una descrizione sommaria del problema del pattern matching e del ruolo dell'informazione negativa in tale ambito di studi.  Nel Capitolo Quattro, si procede ad un'approfondita presentazione dell'algoritmo in esame, tralasciando i dettagli implementativi che verranno affrontati con cura nel Capitolo Cinque. Nel Capitolo Sei, viene presentato a livello generale il tool SMART, definendone gli obiettivi e descrivendone le caratteristiche. Il Capitolo Sette rappresenta il risultato originale di questo lavoro di ricerca, fornendo dettagli circa l'integrazione dell'algoritmo in esame nel tool SMART, delinando gli aspetti implementativi e teorici della soluzione ottenuta. L'ultimo Capitolo consiste in una serie di considerazioni conclusive su lavoro svolto. Infine, viene presentata la bibliografia impiegata per la stesura di questa tesi di laurea.
        \section{Nozioni matematiche sui confronti fra stringhe}
        \section{Informazioni negative e pattern matching}
        \include{contents/chapters/algo}
        \section{Dettagli implementativi dell'algoritmo}
        \section{Overview del tool SMART}
        \include{contents/chapters/final}
        \include{contents/chapters/biblio}

\end{document}