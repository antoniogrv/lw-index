\section{Introduzione}

[..]

Questo primo capitolo consiste esclusivamente in una sintetica introduzione del lavoro svolto. Nel Capitolo Due, vengono presentate le componentistiche matematiche, con opportuni riferimenti accademici, utilizzate dall'algoritmo di confronto fra stringhe in esame, o dal tool SMART. Nel Capitolo Tre, si fornisce - a linee generali - una descrizione sommaria del problema del pattern matching e del ruolo dell'informazione negativa in tale ambito di studi.  Nel Capitolo Quattro, si procede ad un'approfondita presentazione dell'algoritmo in esame, tralasciando i dettagli implementativi che verranno affrontati con cura nel Capitolo Cinque. Nel Capitolo Sei, viene presentato a livello generale il tool SMART, definendone gli obiettivi e descrivendone le caratteristiche. Il Capitolo Sette rappresenta il risultato originale di questo lavoro di ricerca, fornendo dettagli circa l'integrazione dell'algoritmo in esame nel tool SMART, delinando gli aspetti implementativi e teorici della soluzione ottenuta. L'ultimo Capitolo consiste in una serie di considerazioni conclusive su lavoro svolto. Infine, viene presentata la bibliografia impiegata per la stesura di questa tesi di laurea.